\documentclass[12pt,a4paper,oneside]{article}
% -----------------------------------------------------------------------------------------
\usepackage[utf8]{inputenc}
\usepackage[centertags]{amsmath}
\usepackage{amsfonts}
\usepackage{amssymb}
\usepackage{amsthm}
\usepackage{newlfont}
\usepackage{amsxtra}
\usepackage{amstext}
\usepackage{latexsym}
\usepackage[ngerman]{babel}
\usepackage[top=2cm,bottom=2cm,left=2cm,right=2cm]{geometry}
\usepackage{pifont}
\usepackage{ulem}
\usepackage{color}
\usepackage{marvosym}
\usepackage[T1]{fontenc}
\usepackage{amsfonts}
\usepackage{graphicx}
\usepackage{array}
\usepackage{booktabs}
\usepackage{dcolumn}
\usepackage{units}
\usepackage{rotating}
\usepackage{hyperref}
\usepackage{floatflt}
%\usepackage{makeidx}
\usepackage{fancyhdr}
\usepackage{totpages}
\usepackage{amsmath}
\usepackage{framed}
\usepackage{multicol}
\usepackage{lmodern}

\usepackage{mathtools}
% -----------------------------------------------------------------------------------------
\begin{document}
	% -----------------------------------------------------------------------------------------
	%	 Titel
	% -----------------------------------------------------------------------------------------
	%\headheight0cm\headsep0cm\topskip0cm\footskip0cm
	\pagestyle{fancy} %eigener Seitenstil
	\fancyhf{} %alle Kopf- und Fußzeilenfelder bereinigen
	\fancyhead[L]{Formelsammlung Photonik} %Kopfzeile links
	\fancyhead[C]{FS PHO} %zentrierte Kopfzeile
	\fancyhead[R]{Seite \thepage/\ref{TotPages}} %Kopfzeile rechts
	\renewcommand{\headrulewidth}{0.1mm} %obere Trennlinie
	\headheight1cm
	
	\textheight25cm
	\setlength{\parindent}{0em}
	%\fancyfoot[C]{\thepage} %Seitennummer
	%\renewcommand{\footrulewidth}{0.4pt} %untere Trennlinie
	
	\DeclarePairedDelimiter{\abs}{\lvert}{\rvert}
	
	\setlength{\abovedisplayskip}{7pt}
	\setlength{\belowdisplayskip}{7pt}
	~
	\begin{center}
		\vspace{-1.0cm}
		{\Large \textbf{Formelsammlung Photonik}}
	\end{center}
	% -----------------------------------------------------------------------------------------
	%	Hauptext
	% -----------------------------------------------------------------------------------------
	\setlength{\columnseprule}{0.1mm}
	\begin{multicols}{3}
		\textbf{Wellenlänge} $$\lambda=\frac{c_0}{f\, n} = \frac{\lambda_0}{n}\quad[m]$$
		\textbf{Wellenzahl} $$\nu = \frac{1}{\lambda_0}$$
		\textbf{Feldwellenwiderstand} $$Z_F=\frac{|E|}{|H|}=\sqrt{\frac{\mu_0 \, \mu_r}{\epsilon_0 \, \epsilon_r}}$$ $$= Z_0 \, \sqrt{\frac{\mu_r}{\epsilon_r}} \qquad[\Omega]$$
		\textbf{Im Medium} $$Z_F = \frac{Z_0}{n}$$
		\textbf{Poynting-Vektor} $$\vec{S}=\vec{E}\times\vec{H}$$ $$|\vec{S}|=\frac{1}{2}|\vec{E}|\,|\vec{H}|$$$$=\frac{|\vec{E}|^2}{2\,Z_F}=I\quad [\frac{W}{m^2}]$$
		\textbf{Leistung} $$P= A\, |\vec{S}|\quad [W]$$
		\textbf{Photonenenergie} $$W_{Phot}= h\,f=h\, \frac{c_0}{\lambda_0}\quad [J]$$
		\textbf{Photonenflussdichte} $$\Phi_{Phot}=\frac{N_{Phot}}{dt\,dA}$$ $$ = \frac{I}{W_{Phot}} \quad [\frac{1}{m^2\, s}]$$ 
		\textbf{Photonenfluss} $$F_{Phot} = \frac{N_{Phot }}{dt}= \Phi_{Phot} \, A \quad [\frac{1}{s}]	$$
 	\textbf{Snelluissches Brechungsgesetz}$$n_1 \, \sin(\alpha_1) = n_2 \sin(\alpha_2)$$\begin{tiny}
		$n_1$ einfallender, $n_2$ transmittierter Strahl
	\end{tiny}
	\textbf{Fresnelsches Brechungsgesetz }$$n = \frac{n_2 }{n_1}$$\textbf{Senkrechte Polarisation}$$R_s(\alpha, n)$$ $$= \bigg[\frac{\sqrt{n^2 - \sin^2(\alpha)}-\cos{\alpha}}{\sqrt{n^2 - \sin^2(\alpha)}+cos{\alpha}}\bigg]^2$$ $$T_s(\alpha, n) = 1- R_s(\alpha, n)$$
	\textbf{Parallele Polarisation} $$R_p(\alpha, n) $$ $$= \bigg[\frac{n \cos(\alpha)-\sqrt{1-(\frac{\sin(\alpha)}{n})^2}}{n \cos(\alpha)+\sqrt{1-(\frac{\sin(\alpha)}{n}})^2}\bigg]^2$$ $$T_p(\alpha, n) = 1 - R_p(\alpha, n)$$
	\textbf{Senkrechter Einfall}
	$$R = \big( \frac{n_1 - n_2}{n_1 + n_2}\big)^2 = \big( \frac{1- n }{1 + n}\big)^2$$
	\textbf{Totalreflexion} \begin{tiny}\\nur bei dicht $\rightarrow$ duenn
	\end{tiny}$$ \alpha_T = \sin^{-1}(\frac{n_2}{n_1})$$
	\textbf{Brewster-Winkel }\begin{tiny}\\Refl. par. Komp. wird 0, beide Richtungen, \\ 90 Grad zw. Refl. u. Trans.
	\end{tiny}$$\alpha_B = \tan^{-1}(\frac{n_2}{n_1})$$
\textbf{Jones-Vektoren}
$$\vec{J}=\frac{1}{\sqrt{|\hat{E}|^2_x+|\hat{E}^2_y|}} \begin{pmatrix}
\hat{E}_x \\ \hat{E}_y
\end{pmatrix}$$
$$\vec{J}_{h, \alpha}  = \begin{pmatrix}
\cos(\alpha) \\ \sin(\alpha)
\end{pmatrix} $$ 
\begin{tiny} $\alpha$ Winkel z. x-Achse
\end{tiny}
\\Vertikaler Linearpolarisator $$ J_v = \begin{pmatrix}
0  & 0 \\ 0 & 1
\end{pmatrix}$$
\\Horizontaler Linearpolarisator $$ J_v = \begin{pmatrix}
1  & 0 \\ 0 & 0
\end{pmatrix}$$
$$\vec{J}_{nachher}= J_x \cdot \vec{J}_{vorher}$$
Normierte Leistung $$ P \sim |\vec{J}|^2$$
\textbf{AR-Spiegel}\\
Beschichtungsmaterial
$$n_{AR}= \sqrt{n_1 n_2} $$
Dicke $$d_{AR}= \frac{\lambda_{AR}}{4}= \frac{\lambda_0}{4 n_{AR}}$$
\textbf{Vibrationsenergie} $$W_{vib}=h f_{vib} (\nu + \frac{1}{2}) \quad [J, eV]$$\begin{tiny}$\nu = 0, 1, 2... $, aequidistant\\
\end{tiny} 
\textbf{Rotationsenergie} $$W_{rot}= B J (J+1)\quad [J, eV]$$ \begin{tiny}$J=0,1,2...$, nicht aequidistant \\B: Molekuelspez. Konst.\\
\end{tiny}
\textbf{Besetzungsdichte} $$N_\nu = \frac{Anzahl \mu S}{Volumen} \quad [\frac{1}{cm^3}]$$
\textbf{Gesamtbesetzungsdichte} $$N_g = \sum_\nu N_nu$$\\
\textbf{Boltzmann-Verteilung }$$\frac{N_\nu}{N_\mu}=e^{-(\frac{W_\nu-W_\mu}{kT})}$$
\textbf{Absolutwerte} \begin{tiny}\\Mit $W_1 = 0$
\end{tiny}$$N_\nu = N_g \frac{e^{-(\frac{W_\nu}{kT})}}{ \sum_{i=1}^{\infty} e^{-(\frac{W_i}{kT})}}$$ $$ =N_g \frac{e^{-(\frac{W_\nu}{kT})}}{ Q(T)} \quad [\frac{1}{cm^3}]$$
\textbf{Zustandssumme} $$Q(T) = \sum_{i=1}^{\infty} e^{-(\frac{W_i}{kT})} $$
\textbf{Quantenwirkungsgrad} $$\nu_q = \frac{W_{LaserPhot}}{W_{PumpPhot}}$$
\textbf{Leistungskleinsignalverstaerkung} $$g_{KS}=\sigma(N_2-N_1)\quad[\frac{1}{m}]$$
\textbf{Wechselwirkungsquerschnitt} $$\sigma= \frac{\lambda_L^2}{8 \pi \tau_2}\gamma(f)$$\begin{tiny}$\gamma(f)$: Linienprofilfkt. Form d. Verbreiterung geg.\\
\end{tiny}



\end{multicols}
\end{document} 